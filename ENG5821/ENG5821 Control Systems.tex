\documentclass[11pt]{article}
    \usepackage{caption}
    \usepackage{graphicx}
    \usepackage{mathtools}
    \graphicspath{ {img/} }
    \setlength{\parindent}{0pt}
    \DeclareCaptionType{equ}[][]
    \usepackage[svgnames]{xcolor}
    
    \newcommand*{\plogo}{\fbox{$\mathcal{BM}$}}
        
    \usepackage{PTSerif}
    
    \begin{document} 
        
    \begin{titlepage}
    
        \raggedleft
        
        \vspace*{\baselineskip}
        
        {\Large Bryan Melanson}
        
        \vspace*{0.167\textheight}
        
        \textbf{\LARGE How to Not Fail}\\[\baselineskip]
        
        {\textcolor{Red}{\Huge Control Systems}}\\[\baselineskip]
        
        {\Large \textit{While never going to class}}
        
        \vfill
        
        {\large Computer Engineering 2020 ~~\plogo}
        
        \vspace*{3\baselineskip}
    
    \end{titlepage}

    \pagebreak

    \tableofcontents

    \pagebreak

    \section{Modeling in the Frequency Domain}

    \section{Modeling in the Time Domain}
    \section{Time Response}
    \section{Reduction of Multiple Systems}
    \section{Stability}
    \subsection{Routh-Hurwitz Criteria}
    \subsection{Routh-Hurwitz Special Cases}
    \section{Steady State Errors}
    \section{Root Locus Techniques}
    
    The root locus approaches straight lines as asymptotes as the locus approaches infinity. Further, the equation of the asymptotes is given by the real-axis intercept, $\sigma_a$ and angle, $\theta_a$ as follows:
    \begin{center}
        $\sigma_a = \frac{\Sigma \text{ Finite Poles} - \Sigma \text{ Finite Zeros}}{\text{\# Finite Poles - \# Finite Zeros}}$ 
    \end{center}

    \begin{center}
        $\theta_a = \frac{(2k - 1)\pi}{\text{\# Finite Poles - \# Finite Zeros}}$ 
    \end{center}

    Where $k = \pm 0, \pm 1, \pm 2, \pm 3$ and the angle is given in radians with respect to the positive
extension of the real axis.
    \section{Design via Root Locus}
    \section{Frequency Response Techniques}
    \section{Design via Frequency Response}
    
    \end{document}