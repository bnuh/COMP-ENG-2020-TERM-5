\documentclass[11pt]{article}
    \usepackage{caption}
    \setlength{\parindent}{0pt}
    \DeclareCaptionType{equ}[][]
    \usepackage[svgnames]{xcolor}
    
    \newcommand*{\plogo}{\fbox{$\mathcal{BM}$}}
    
    \usepackage{PTSerif}
    
    \begin{document} 
        
    \begin{titlepage}
    
        \raggedleft
        
        \vspace*{\baselineskip}
        
        {\Large Bryan Melanson}
        
        \vspace*{0.167\textheight}
        
        \textbf{\LARGE How I Learned to Not Fail}\\[\baselineskip]
        
        {\textcolor{Red}{\Huge Probability \& Random Processes}}\\[\baselineskip]
        
        {\Large \textit{While never going to class}}
        
        \vfill
        
        {\large Computer Engineering 2020 ~~\plogo}
        
        \vspace*{3\baselineskip}
    
    \end{titlepage}

    \pagebreak
    
%%%%%%%%%%%%%%%%%%%%%%%%%%%%%%%%%%%%%%%%%%%%%%%%%

    \tableofcontents

    \pagebreak

%%%%%%%%%%%%%%%%%%%%%%%%%%%%%%%%%%%%%%%%%%%%%%%%%

    \section{Descriptive Statistics}

    \subsection{Statistics}
    Statistics are the summarization of a set of data that has been collected, which demonstrates random variation. \textit{Extracting meaning from data.}

    \subsection{Inferential Statistics}
    Making inferences abotu a situation based on data, such as forecasting. \\
    Descriptive statistics can be the basis for inferences.

    \subsection{Representative Values}
    \begin{enumerate}
        \item{Mean} 
        \item{Median}
        \item{Mode}
        \item{Range \textit{- [Min, Max]}}
        \item{Variance \textit{- Average of deviation squared from the mean}}
        \item{Standard Deviation \textit{- Measure of average absolute deviation}}
        \item{Skewness \textit{- Measure of the shape of the distribution funciton}}
        \item{Quantiles \textit{- Generalizatino of the median to percentiles}}
    \end{enumerate}
    \subsection{Observational vs. Experimental Data}
    Experimental involves manipulation objects to determine cause and effect in data. Observational refers to naturally occurring events.
    \pagebreak

%%%%%%%%%%%%%%%%%%%%%%%%%%%%%%%%%%%%%%%%%%%%%%%%%

    \section{Basic Probability}
    \subsection{Probability Calculus}
    Probability events have a total probability between zero and one.
        \begin{equ}[!ht]
            \begin{equation}
              Pr[An Event] = 1
            \end{equation}
          \caption{An event which is sure to happen}
        \end{equ} 

The definition of probability for how often an event is observed can be related to the number of repetiions of the experiment. \\


\begin{equ}[!ht]
    \begin{equation}
      Pr[Heads] =  \frac{\textrm{number } k \textrm{ of Heads in }N \textrm{ coin tosses}}{\textrm{ coin tosses}}
    \end{equation}
  \caption{Counting the probability of heads in a set of coin tosses}
\end{equ} 

The larger the number of repetiions, the higher accuracy with which we can predict the likelihood of an event happening.

\subsection{Probability Model}

\subsubsection{Events}
Events are elements in the set of possible outcomes in an experiment.

\subsubsection{Sample Space}
The set of all possible outcomes for an experiment.

\begin{equ}[!ht]
    \begin{equation}
      S = \{1,2,3,4,5,6\}
    \end{equation}
  \caption{The sample space for a dice roll}
\end{equ} 

\begin{equ}[!ht]
    \begin{equation}
      A_1 = \{1\}, A_2 = \{1,2,5\}
    \end{equation}
  \caption{Subsets containing events in the sample space}
\end{equ} 



%%%%%%%%%%%%%%%%%%%%%%%%%%%%%%%%%%%%%%%%%%%%%%%%%

    \end{document}